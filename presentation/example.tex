\section{\ttbarH Analysis Study}
\subsection{Input for Analysis}
\begin{frame}{Parameters}
\begin{block}{Categories and Processes used}
\begin{itemize}
\item For $\text{j}\geq 6,\quad \text{t}\geq 4$: all Processes\footnote{Full list with all considered processes can be found in the backup\\}; \ttbarH \bbbar + \ttbar\bbbar\\
\item For $\text{j}\geq 6,\quad \text{t} = 3$: all Processes
\end{itemize}
\end{block}

\begin{itemize}
\item In every case the datacards were checked by fitting an asimov toy set to make sure they were intact
\item The following slides show the analysis of processes \ttbarH \bbbar + \ttbar\bbbar in category $\text{j}\geq 6,\quad \text{t}\geq 4$ where only the nuisance parameter for luminosity\footnote{Full list with all considered nuisance parameters can be found in the backup} was considered as an example
\end{itemize}

%\parnotes
\end{frame}
\subsection{Example: $\text{j}\geq 6,\quad \text{t}\geq 4$, \ttbarH \bbbar + \ttbar\bbbar}

\begin{frame}{Generate Pseudo Data}
\begin{figure}
\centering
\subfloat[\ttbarH \bbbar]{\includegraphics[scale=0.27]{plots/tthbb_finaldiscr_jge6_tge4.pdf}}$\qquad$
\subfloat[\ttbar\bbbar]{\includegraphics[scale=0.27]{plots/ttbb_finaldiscr_jge6_tge4.pdf}}

\end{figure}
\vskip -0.3cm
\begin{block}{How to Generate Pseudo Data}
\begin{itemize}
\item add histograms bin-wise
\item use each sum respectively as the expected value of a Poisson distribution
\item get a random number from the corresponding distribution and fill the Pseudo Data histogram bin with it
\item create new datacard for each Pseudo Experiment
\end{itemize}
\end{block}
\end{frame}

\begin{frame}{Example Pseudo Data Distribution}
\begin{figure}
\centering
\includegraphics[scale=0.4]{plots/data_obs_finaldiscr_ljets_jge6_tge4_bin4.pdf}
\caption[Example: $\text{j}\geq 6,\quad \text{t}\geq 4$, \ttbarH \bbbar + \ttbar\bbbar Pseudo Data Histogram Bin 4]{Example: $\text{j}\geq 6,\quad \text{t}\geq 4$, \ttbarH \bbbar + \ttbar\bbbar Pseudo Data Histogram Bin 4}
\end{figure}
\end{frame}

\begin{frame}{Fit Parameters for Pseudo Data}
\begin{block}{Requirements for Fits}
\begin{itemize}
\item \texttt{fit$\_$status} flags in RooFit Objects have to be 0\\
\item covariance matrix has to be accurate (\texttt{fit$\_$b/s::covQual() == 3})
\end{itemize}

\end{block}

\begin{block}{\texttt{combine} commands}
\begin{itemize}
\item \texttt{combine -M MaxLikelihoodFit -m 125 \-\-minimizerStrategy 0 --minimizerTolerance 0.1 --saveNormalizations --saveShapes --rMin=-10 --rMax=10}
\item also tried option \texttt{--minimizerTolerance 0.001}, no visible effects
\item also tried reduced fit range \texttt{--rMin = -5 --rMax = 5}, no visible effects
%\item generated asimov toy set for MC datacard with option \texttt{-t -1 --expectSignal 0OR1} to make sure datacard is not damaged

\end{itemize}
\end{block}
\end{frame}

\begin{frame}{Fit Results for $\text{j}\geq 6,\text{t}\geq 4$, \ttbarH \bbbar + \ttbar\bbbar}
\vskip-0.5cm
\begin{figure}
\subfloat[Fitted Signal Strength Distribution\label{fig::fittedMuDistr}]{\includegraphics[scale=0.23]{plots/jge6_tge4_ttHbb_ttbb_N100_sig0_lumiOnly_POI.pdf}}$\quad$
\subfloat[Mean Fitted Signal Strength Distribution\label{fig::meanMuDistr}]{\includegraphics[scale=0.23]{plots/jge6_tge4_ttHbb_ttbb_N100_sig0_lumiOnly_POImeans.pdf}}
\caption[Distributions for Fit Results for $\text{j}\geq 6,\text{t}\geq 4$, \ttbarH \bbbar + \ttbar\bbbar of 100 Pseudo Data with nominal signal strength $\mu=0$]{Distributions for Fit Results for $\text{j}\geq 6,\text{t}\geq 4$, \ttbarH \bbbar + \ttbar\bbbar of 100 Pseudo Data with nominal signal strength $\mu=0$.
\ref{fig::fittedMuDistr} shows the $\mu$ distribution obtained for each Pseudo Experiment.
\ref{fig::meanMuDistr} depicts the mean value and error (circle with full line) and the median with the RMS (square with vertical dotted line) of \ref{fig::fittedMuDistr}.
The nominal signal strength is shown as a horizontal dotted line. }
\end{figure}
\end{frame}

