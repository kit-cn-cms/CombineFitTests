%% LaTeX-Beamer template for KIT design
%% by Erik Burger, Christian Hammer
%% title picture by Klaus Krogmann
%%
%% version 2.1
%%
%% mostly compatible to KIT corporate design v2.0
%% http://intranet.kit.edu/gestaltungsrichtlinien.php
%%
%% Problems, bugs and comments to
%% burger@kit.edu

\documentclass[18pt]{beamer}
\usepackage[utf8]{inputenc}
\usepackage[T1]{fontenc}
\usepackage{microtype}
%% SLIDE FORMAT

% use 'beamerthemekit' for standard 4:3 ratio
% for widescreen slides (16:9), use 'beamerthemekitwide'

\usepackage{templates/beamerthemekit}
% \usepackage{templates/beamerthemekitwide}
\usepackage{amsmath}
\usepackage{amsfonts}
\usepackage{amssymb}
\usepackage{siunitx}
\sisetup{range-units = single, range-phrase = {-}, separate-uncertainty = true,multi-part-units=brackets, product-units=brackets, locale=US, detect-weight=true, binary-units=true}
\usepackage{color}
\usepackage{todonotes}
\presetkeys{todonotes}{inline}{}
\usepackage{booktabs}
\usepackage{listings}
\usepackage{parnotes}
\usepackage{xspace}
\usepackage{multirow}
\usepackage{subfig}
\usepackage{caption}
\captionsetup{font=scriptsize,labelfont=scriptsize}

\usepackage{chngcntr}
\counterwithin{figure}{section}
\counterwithin{table}{section}
\setbeamertemplate{caption}[numbered]

% settings for making the table of contents less ugly
\setbeamertemplate{section in toc}{\inserttocsectionnumber\ \ ~\inserttocsection}
\setbeamertemplate{subsection in toc}{\qquad\inserttocsectionnumber.\inserttocsubsectionnumber\ \ ~\inserttocsubsection\\ }

% bibliography
%\usepackage[style=numeric, backend=biber, sorting=ynt]{biblatex}
%\addbibresource{bibliography.bib}
\beamertemplatenavigationsymbolsempty
%% TITLE PICTURE

% if a custom picture is to be used on the title page, copy it into the 'logos'
% directory, in the line below, replace 'mypicture' with the 
% filename (without extension) and uncomment the following line
% (picture proportions: 63 : 20 for standard, 169 : 40 for wide
% *.eps format if you use latex+dvips+ps2pdf, 
% *.jpg/*.png/*.pdf if you use pdflatex)

\titleimage{Karlsruhe_00315u}


%% TITLE LOGO

% for a custom logo on the front page, copy your file into the 'logos'
% directory, insert the filename in the line below and uncomment it

\titlelogo{nologo.png}

% (*.eps format if you use latex+dvips+ps2pdf,
% *.jpg/*.png/*.pdf if you use pdflatex)

%% TikZ INTEGRATION

% use these packages for PCM symbols and UML classes
% \usepackage{templates/tikzkit}
% \usepackage{templates/tikzuml}

% Define KIT Colors
\definecolor{KIT-Gruen}{RGB}{0,150,130}
\definecolor{KIT-Blau}{RGB}{70,100,170}
\definecolor{KIT-MaiGruen}{RGB}{140,182,60}
\definecolor{KIT-Gelb}{RGB}{252,229,0}
\definecolor{KIT-Orange}{RGB}{223,155,27}
\definecolor{KIT-Braun}{RGB}{167,130,46}
\definecolor{KIT-Rot}{RGB}{162,34,35}
\definecolor{KIT-Lila}{RGB}{163,16,124}
\definecolor{KIT-Cyan-Blau}{RGB}{35,161,224}


% Define some commands
% arrows
\newcommand{\rar}{\ensuremath{\rightarrow}\xspace}
\newcommand{\Rar}{\ensuremath{\Rightarrow}\xspace}

% particle physics
\newcommand{\ttbar}{\text{t}\ensuremath{\overline{\text{t}}}\xspace }
\newcommand{\boldttbar}{\textbf{t}\ensuremath{\overline{\textbf{t}}}\xspace}
\newcommand{\ttbarH}{\text{t}\ensuremath{\overline{\text{t}}}H\xspace }
\newcommand{\boldttbarH}{\textbf{t}\ensuremath{\overline{\textbf{t}}}\textbf{H}\xspace}
\newcommand{\bbbar}{\text{b}\ensuremath{\overline{\text{b}}}\xspace }
\newcommand{\ccbar}{\text{c}\ensuremath{\overline{\text{c}}}\xspace }
\newcommand{\pT}{\ensuremath{{p_{T}}}\xspace}
\newcommand{\Htobb}{H\ensuremath{\to } \bbbar}

\newcommand{\pathToPlots}{../plots}
\newcommand{\pathToTables}{../tables}

\newcommand{\catname}{jge6_t3}
\newcommand{\frameCatName}{63}
\newcommand{\scaledProcessName}{63445464_ttHbb_N1000_ttbarOther_0p9}
\newcommand{\frameScaledProcessName}{$\ttbar\cdot 0.9$}


\newcommand{\createNormTableFrame}[7]{

\begin{frame}{Normalization for #4 - #2 - Values for $\mu = #6$}
\begin{scriptsize}
\input{#1/63445464_ttHbb_N1000_#3_ljets_#5_normalisation_values_nominal_S_#7}
\end{scriptsize}

\end{frame}
}

\newcommand{\createPOItables}[5]{
%#1: path/to/tables
%#2: additional POI(s) in path form
%#3: additional POI(s) in text form
%#4: process scaling in path form
%#5: process scaling in text form


\begin{frame}{#3, #5 - Asimov POI values}

\begin{scriptsize}
\input{#1/asimov/#2/63445464_ttHbb_N1000_#2_#4_POI.tex}
\end{scriptsize}

\end{frame}

\begin{frame}{#3 - #5 - PseudoData POI values}

\begin{scriptsize}
\input{#1/PseudoData/#2/63445464_ttHbb_N1000_#2_#4_POI.tex}
\end{scriptsize}

\end{frame}

}

\newcommand{\createPullplotFrame}[3]{
%#1: path/to/plots
%#2: process scaling in text form
%#3: process scaling in path form
\begin{frame}{#2 - 63-44-54-64 - Nuisance Parameter Pullplots}

\begin{figure}

\begin{minipage}{0.48\textwidth}
\centering
\includegraphics[width=\textwidth]{#1/63445464_ttHbb_N1000_#3_pullplot_nominal_S_0p0.pdf}
\caption{Pull Plot for $\mu = 0$ in Category 63-44-54-64 with #2}

\end{minipage}
\hfill
\begin{minipage}{0.48\textwidth}
\centering
\includegraphics[width=\textwidth]{#1/63445464_ttHbb_N1000_#3_pullplot_nominal_S_1p0.pdf}
\caption{Pull Plot for $\mu = 1$ in Category 63-44-54-64 with #2}

\end{minipage}

\end{figure}
\end{frame}
}

\newcommand{\createPullplotFrameAsimovComparison}[7]{
%#1: path/to/plots
%#2: additional POI(s) in path form
%#3: additional POI(s) in text form
%#4: process scaling in path form
%#5: process scaling in text form
%#6: fit + injected signal strength in path form
%#7: fit + injected signal strength in text form


\begin{frame}{#3, #5 - 63-44-54-64 - Pull Plots for #7}

\begin{minipage}{0.44\textwidth}
\centering
\includegraphics[width=\textwidth]{#1/asimov/#2/63445464_ttHbb_N1000_#2_#4_pullplot_#6.pdf}\\
\vskip -0.2cm
Asimov Pull Plot of #7

\end{minipage}
\hfill
\begin{minipage}{0.44\textwidth}
\centering
\includegraphics[width=\textwidth]{#1/PseudoData/#2/63445464_ttHbb_N1000_#2_#4_pullplot_#6.pdf}\\
\vskip -0.2cm
PseudoData Pull Plot of #7

\end{minipage}

\end{frame}


}

\newcommand{\createCorrelationplotFrameAsimovComparison}[7]{
%#1: path/to/plots
%#2: additional POI(s) in path form
%#3: additional POI(s) in text form
%#4: process scaling in path form
%#5: process scaling in text form
%#6: fit + injected signal strength in path form
%#7: fit + injected signal strength in text form


\begin{frame}{#3, #5 - 63-44-54-64 - Correlation PostfitB for #7}

\begin{minipage}{0.44\textwidth}
\centering
\includegraphics[width=\textwidth]{#1/asimov/#2/63445464_ttHbb_N1000_#2_#4_correlationPlot_PostfitB_#6.pdf}\\
\vskip -0.2cm
Asimov Correlation Plot of #7

\end{minipage}
\hfill
\begin{minipage}{0.44\textwidth}
\centering
\includegraphics[width=\textwidth]{#1/PseudoData/#2/63445464_ttHbb_N1000_#2_#4_correlationPlot_PostfitB_#6.pdf}\\
\vskip -0.2cm
PseudoData Correlation Plot of #7

\end{minipage}

\end{frame}


\begin{frame}{#3, #5 - 63-44-54-64 - Correlation PostfitS for #7}

\begin{minipage}{0.44\textwidth}
\centering
\includegraphics[width=\textwidth]{#1/asimov/#2/63445464_ttHbb_N1000_#2_#4_correlationPlot_PostfitS_#6.pdf}\\
\vskip -0.2cm
Asimov Correlation Plot of #7

\end{minipage}
\hfill
\begin{minipage}{0.44\textwidth}
\centering
\includegraphics[width=\textwidth]{#1/PseudoData/#2/63445464_ttHbb_N1000_#2_#4_correlationPlot_PostfitS_#6.pdf}\\
\vskip -0.2cm
PseudoData Correlation Plot of #7

\end{minipage}

\end{frame}

}

\newcommand{\createNormPullplotFrame}[5]{
%#1: path/to/plots
%#2: process scaling in text form
%#3: process scaling in path form
%#4: category in text form
%#5: category in path form
\begin{frame}{#2 - #4 - Normalization Pullplots}

\begin{figure}

\begin{minipage}{0.48\textwidth}
\centering
\includegraphics[width=\textwidth]{#1/63445464_ttHbb_N1000_#3_ljets_#5_normalisation_pullplot_nominal_S_0p0.pdf}
\caption{Normalization Pull Plot for $\mu = 0$ in Category #4 with #2}

\end{minipage}
\hfill
\begin{minipage}{0.48\textwidth}
\centering
\includegraphics[width=\textwidth]{#1/63445464_ttHbb_N1000_#3_ljets_#5_normalisation_pullplot_nominal_S_1p0.pdf}
\caption{Normalization Pull Plot for $\mu = 1$ in Category #4 with #2}

\end{minipage}

\end{figure}
\end{frame}
}


\newcommand{\createNormPullplotFrameAsimovComparison}[9]{
%#1: path/to/plots
%#2: additional POI(s) in path form
%#3: additional POI(s) in text form
%#4: process scaling in path form
%#5: process scaling in text form
%#6: category in path form
%#7: category in text form
%#8: fit + injected signal strength in path form
%#9: fit + injected signal strength in text form


\begin{frame}{#3, #5 - #7 - Norm Pull Plots for #9}

\begin{minipage}{0.44\textwidth}
\centering
\includegraphics[width=\textwidth]{#1/asimov/#2/63445464_ttHbb_N1000_#2_#4_ljets_#6_normalisation_pullplot_#8.pdf}\\
\vskip -0.2cm
Asimov Pull Plot of #9 in category #7

\end{minipage}
\hfill
\begin{minipage}{0.44\textwidth}
\centering
\includegraphics[width=\textwidth]{#1/PseudoData/#2/63445464_ttHbb_N1000_#2_#4_ljets_#6_normalisation_pullplot_#8.pdf}\\
\vskip -0.2cm
PseudoData Pull Plot of #9 in category #7

\end{minipage}

\end{frame}


}


\newcommand{\createNormFrameSet}[4]{

%#1: path/to/plots
%#2: path/to/tables
%#3: process scaling in text form
%#4: process scaling in path form

\createPullplotFrame{#1}{#3}{#4}
\createNormPullplotFrame{#1}{#3}{#4}{44}{j4_t4}
\createNormTableFrame{#2}{#2}{#3}{44}{j4_t4}{0.0}{0p0}
\createNormTableFrame{#2}{#2}{#3}{44}{j4_t4}{1.0}{1p0}

\createNormPullplotFrame{#1}{#3}{#4}{54}{j5_tge4}
\createNormTableFrame{#2}{#2}{#3}{54}{j5_tge4}{0.0}{0p0}
\createNormTableFrame{#2}{#2}{#3}{54}{j5_tge4}{1.0}{1p0}

\createNormPullplotFrame{#1}{#3}{#4}{63}{jge6_t3}
\createNormTableFrame{#2}{#2}{#3}{63}{jge6_t3}{0.0}{0p0}
\createNormTableFrame{#2}{#2}{#3}{63}{jge6_t3}{1.0}{1p0}

\createNormPullplotFrame{#1}{#3}{#4}{64}{jge6_tge4}
\createNormTableFrame{#2}{#2}{#3}{64}{jge6_tge4}{0.0}{0p0}
\createNormTableFrame{#2}{#2}{#3}{64}{jge6_tge4}{1.0}{1p0}


}




\newcommand{\compareToAsimov}[4]{
%#1: additional POI in path form
%#2: additional POI in text form
%#3: process scaling in path form
%#4: process scaling in text form

\begin{frame}
\begin{center}

\begin{huge}
\hypertarget{#1_#3}{\textbf{#2 - #4}}
\end{huge}

\end{center}
\end{frame}


\createPOItables{\pathToTables/JTBDT_Spring17v10/wo_NP}{#1}{#2}{#3}{#4}

\createCorrelationplotFrameAsimovComparison{\pathToPlots/JTBDT_Spring17v10/wo_NP}{#1}{#2}{#3}{#4}{nominal_S_0p0}{r-Fit with injected $\mu = 0$}
\createCorrelationplotFrameAsimovComparison{\pathToPlots/JTBDT_Spring17v10/wo_NP}{#1}{#2}{#3}{#4}{MDFnominal_S_0p0}{MDF with injected $\mu = 0$}
\createCorrelationplotFrameAsimovComparison{\pathToPlots/JTBDT_Spring17v10/wo_NP}{#1}{#2}{#3}{#4}{nominal_S_1p0}{r-Fit with injected $\mu = 1$}
\createCorrelationplotFrameAsimovComparison{\pathToPlots/JTBDT_Spring17v10/wo_NP}{#1}{#2}{#3}{#4}{MDFnominal_S_1p0}{MDF with injected $\mu = 1$}

\createPullplotFrameAsimovComparison{\pathToPlots/JTBDT_Spring17v10/wo_NP}{#1}{#2}{#3}{#4}{nominal_S_0p0}{r-Fit with injected $\mu = 0$}
\createPullplotFrameAsimovComparison{\pathToPlots/JTBDT_Spring17v10/wo_NP}{#1}{#2}{#3}{#4}{MDFnominal_S_0p0}{MDF with injected $\mu = 0$}
\createPullplotFrameAsimovComparison{\pathToPlots/JTBDT_Spring17v10/wo_NP}{#1}{#2}{#3}{#4}{nominal_S_1p0}{r-Fit with injected $\mu = 1$}
\createPullplotFrameAsimovComparison{\pathToPlots/JTBDT_Spring17v10/wo_NP}{#1}{#2}{#3}{#4}{MDFnominal_S_1p0}{MDF with injected $\mu = 1$}

\createNormPullplotFrameAsimovComparison{\pathToPlots/JTBDT_Spring17v10/wo_NP}{#1}{#2}{#3}{#4}{j4_t4}{44}{nominal_S_0p0}{r-Fit with injected $\mu = 0$}
\createNormPullplotFrameAsimovComparison{\pathToPlots/JTBDT_Spring17v10/wo_NP}{#1}{#2}{#3}{#4}{j4_t4}{44}{MDFnominal_S_0p0}{MDF with injected $\mu = 0$}
\createNormPullplotFrameAsimovComparison{\pathToPlots/JTBDT_Spring17v10/wo_NP}{#1}{#2}{#3}{#4}{j4_t4}{44}{nominal_S_1p0}{r-Fit with injected $\mu = 1$}
\createNormPullplotFrameAsimovComparison{\pathToPlots/JTBDT_Spring17v10/wo_NP}{#1}{#2}{#3}{#4}{j4_t4}{44}{MDFnominal_S_1p0}{MDF with injected $\mu = 1$}

\createNormPullplotFrameAsimovComparison{\pathToPlots/JTBDT_Spring17v10/wo_NP}{#1}{#2}{#3}{#4}{j5_tge4}{54}{nominal_S_0p0}{r-Fit with injected $\mu = 0$}
\createNormPullplotFrameAsimovComparison{\pathToPlots/JTBDT_Spring17v10/wo_NP}{#1}{#2}{#3}{#4}{j5_tge4}{54}{MDFnominal_S_0p0}{MDF with injected $\mu = 0$}
\createNormPullplotFrameAsimovComparison{\pathToPlots/JTBDT_Spring17v10/wo_NP}{#1}{#2}{#3}{#4}{j5_tge4}{54}{nominal_S_1p0}{r-Fit with injected $\mu = 1$}
\createNormPullplotFrameAsimovComparison{\pathToPlots/JTBDT_Spring17v10/wo_NP}{#1}{#2}{#3}{#4}{j5_tge4}{54}{MDFnominal_S_1p0}{MDF with injected $\mu = 1$}

\createNormPullplotFrameAsimovComparison{\pathToPlots/JTBDT_Spring17v10/wo_NP}{#1}{#2}{#3}{#4}{jge6_t3}{63}{nominal_S_0p0}{r-Fit with injected $\mu = 0$}
\createNormPullplotFrameAsimovComparison{\pathToPlots/JTBDT_Spring17v10/wo_NP}{#1}{#2}{#3}{#4}{jge6_t3}{63}{MDFnominal_S_0p0}{MDF with injected $\mu = 0$}
\createNormPullplotFrameAsimovComparison{\pathToPlots/JTBDT_Spring17v10/wo_NP}{#1}{#2}{#3}{#4}{jge6_t3}{63}{nominal_S_1p0}{r-Fit with injected $\mu = 1$}
\createNormPullplotFrameAsimovComparison{\pathToPlots/JTBDT_Spring17v10/wo_NP}{#1}{#2}{#3}{#4}{jge6_t3}{63}{MDFnominal_S_1p0}{MDF with injected $\mu = 1$}

\createNormPullplotFrameAsimovComparison{\pathToPlots/JTBDT_Spring17v10/wo_NP}{#1}{#2}{#3}{#4}{jge6_tge4}{64}{nominal_S_0p0}{r-Fit with injected $\mu = 0$}
\createNormPullplotFrameAsimovComparison{\pathToPlots/JTBDT_Spring17v10/wo_NP}{#1}{#2}{#3}{#4}{jge6_tge4}{64}{MDFnominal_S_0p0}{MDF with injected $\mu = 0$}
\createNormPullplotFrameAsimovComparison{\pathToPlots/JTBDT_Spring17v10/wo_NP}{#1}{#2}{#3}{#4}{jge6_tge4}{64}{nominal_S_1p0}{r-Fit with injected $\mu = 1$}
\createNormPullplotFrameAsimovComparison{\pathToPlots/JTBDT_Spring17v10/wo_NP}{#1}{#2}{#3}{#4}{jge6_tge4}{64}{MDFnominal_S_1p0}{MDF with injected $\mu = 1$}
}





% the presentation starts here

\title[\ttbarH Analysis Study]{\ttbarH Fit Model Stability Studies}
\subtitle{Status Update}
\author[Philip Keicher]{Karim~El~Morabit, Matthias~Schröder, \textbf{Philip~Keicher}}

\institute{Institut für Experimentelle Teilchenphysik (ETP)}


\begin{document}

% change the following line to "ngerman" for German style date and logos
%\selectlanguage{ngerman}
\selectlanguage{english}


%title page
\begin{frame}
\titlepage
\end{frame}
\setcounter{tocdepth}{2}
\section*{Outline}
%\subsection*{Outline}
\begin{frame}[label={outline}]
	\frametitle{Outline}
	\tableofcontents
\end{frame}


\section{Introduction}
\subsection{Goals for the Study}
\begin{frame}{Goals}
\begin{block}{Perform Test of JES Uncertainty Interpolation/Extrapolation}
\begin{itemize}
\item Prepare data for differently scaled JES uncertainties\\
\item Perform Maximum Likelihood Fit\\
\rar How do the constraints change?\\
\rar How are the other uncertainties pulled?\\
\rar How do the normalizations of the different process templates change?

\end{itemize}
\end{block}

\end{frame}

\subsection{Setup}
\begin{frame}{Setup}
\begin{itemize}
\item CMSSW version 7.4.7\\
\end{itemize}
\begin{block}{\texttt{combine} Command for Toy Generation}
combine -M GenerateOnly -m 125 --saveToys -t -1 -n \_-1toys\_sig1 --expectSignal 1 -s 123456 \$toyDatacard
\end{block}
\begin{block}{\texttt{combine} Command for Fit}
combine -M MaxLikelihoodFit -m 125 --minimizerStrategy 0 --minimizerTolerance 0.001 --saveNormalizations --saveShapes --rMin=-10.00 --rMax=10.00 --floatAllNuisances 1 -t -1 --toysFile \$toyFile --minos all \$targetDatacard
\end{block}
\end{frame}

\begin{frame}{List of Processes}
\lstinputlisting[numbers=none, basicstyle=\tiny,  frame=none]{\pathToTables/listOfProcesses.txt}
\end{frame}
\begin{frame}{List of Uncertainties}
\lstinputlisting[numbers=none, basicstyle=\tiny,  frame=none, multicols=2]{\pathToTables/listOfNP_64.txt}
\end{frame}
\sisetup{round-mode=places,round-precision=3, scientific-notation=fixed, fixed-exponent=0}
\section{Analysis Strategy}
\begin{frame}
\centering
\begin{huge}
\textbf{Analysis Strategy}
\end{huge}

\end{frame}


\subsection{Input for Analysis}

\begin{frame}{Input for Analysis}

\begin{block}{Input Parameters}
\begin{itemize}
\item datacard: Spring17v10 \href{https://indico.cern.ch/event/628833/contributions/2624507/attachments/1474840/2283855/KITv10p2.pdf}{\textcolor{blue}{go to presentation}}
\item default categories: $j\geq 6, t=3$; $j=4, t=4$; $j=5, t\geq 4$; $j\geq 6, t\geq 4$ (63-44-54-64)
\item signal process: \ttbarH(\bbbar)
\item background processes: \ttbar+lf, \ttbar+b, \ttbar+2b, \ttbar+\bbbar, \ttbar + \ccbar
%\item only considering b-tagging, cross section, lumi and lepton efficiency uncertainties, no JES uncertainties\\
\item list of considered uncertainties can be found on the next slide
\end{itemize}
\end{block}
\begin{itemize}
\item Following example shows results for combined fit in default categories with $\num[round-precision=1]{1.2}\cdot$(\ttbar+\bbbar)
\end{itemize}

\end{frame}

\begin{frame}{List of Nuisance Parameters}

\lstinputlisting[numbers=none, basicstyle=\tiny,  frame=none]{../listOfNP.txt}
\end{frame}

\subsection{Analysis Steps}

\subsubsection{Input Histograms}
\begin{frame}{Templates for Category 64}

\begin{minipage}{0.48\textwidth}
\begin{center}
\includegraphics[width=\textwidth]{\pathToPlots/JTBDT_Spring17v10/ljets_jge6_tge4_JTBDT_Spring17v10_original_stackplot.pdf}\label{fig::stackPlot64}\\
Stacked Plot of BDT Template Histograms

\end{center}
\end{minipage}
\begin{minipage}{0.48\textwidth}
\begin{center}
\includegraphics[width=\textwidth]{\pathToPlots/JTBDT_Spring17v10/ljets_jge6_tge4_JTBDT_Spring17v10_original_normed_stackplot.pdf}\label{fig::normed_stackPlot64}\\
Normalized BDT Template Histograms

\end{center}

\end{minipage}



\end{frame}
\subsubsection{Workflow}
\begin{frame}{Toy Generation}

%
%\subsubsection{Step 1: Scale Histograms}
%\begin{frame}{Step 1: Scale Histograms}
%
%\begin{minipage}{0.48\textwidth}
%\begin{center}
%\includegraphics[width=\textwidth]{\pathToPlots/JTBDT_Spring17v10/ljets_jge6_tge4_JTBDT_Spring17v10_original_stackplot.pdf}\\
%Stacked Plot of Original BDT Template Histograms
%\label{fig::compare_stackPlot64}
%\end{center}
%
%\end{minipage}
%\begin{minipage}{0.48\textwidth}
%\begin{center}
%\includegraphics[width=\textwidth]{\pathToPlots/JTBDT_Spring17v10/ljets_jge6_tge4_JTBDT_Spring17v10_ttHbb_scaled_ttbarPlusBBbar_1p2_stackplot.pdf}\\
%Stacked Plot of BDT Template Histograms with \ttbar+\bbbar Scaled by \num[round-precision=1]{1.2}
%\label{fig::compare_scaled_stackPlot64}
%\end{center}
%
%\end{minipage}
%
%
%
%
%\end{frame}
%
%\subsubsection{Step 2: Generate Toy Data}
%\begin{frame}{Step 2: Generate Toy Data}
%\begin{block}{Generate Toys from Scaled BDT Template}
%\begin{itemize}
%\item Sample toy measurement from Poisson distribution around bin mean
%\item Use different injected signal strengths for toy generation (nominal signal strength)
%\item Systematic uncertainties considered according to assumed priors when throwing the toys (via using \texttt{combine})
%\item Generate \num{1000} toys for each nominal signal strength
%\end{itemize}
%\end{block}
%\end{frame}
%
%\subsubsection{Step 3: Perform Fit}
%
%\begin{frame}{Step 3: Perform Fit}
%\begin{block}{Fit Original Histograms to Toys from Scaled Data}
%\begin{itemize}
%\item Fit original histograms to generated toy\\
%$\rightarrow$ Perform MaxLikelihoodFit with signal strength as POI via \texttt{combine}\\
%$\rightarrow$ Check distributions
%\end{itemize}
%\end{block}
\begin{block}{Step 1: Scale Process Template}
\begin{itemize}
\item scale specific process template(s) with constant factor\\
\rar create PseudoData corresponding to cross section that differs from expectation
\end{itemize}
\end{block}

\begin{block}{Step 2: Generate Toy Data}
\begin{itemize}
\item Sample toy measurement from Poisson distribution around bin mean
\item Use different injected signal strengths for toy generation (nominal signal strength)
\item Systematic uncertainties considered according to assumed priors when throwing the toys (via using \texttt{combine})
\end{itemize}
\end{block}

\end{frame}

\begin{frame}{Analyzing the Data}

\begin{block}{Step 3: Perform Fit}
\begin{itemize}
\item Fit original templates to generated toys\\
\rar Perform \texttt{MaxLikelihoodFit} with signal strength as POI\\
\rar Perform \texttt{MaxLikelihoodFit} with additional POIs for scaled processes\\
\rar Collect distributions
\end{itemize}
\end{block}

\begin{block}{Collect Fit Results}
\begin{itemize}
\item Analyze distribution of fit results for each parameter, respectively
\item Is the average POI equal to the injected signal strength during toy generation? 
\end{itemize}
\end{block}

\end{frame}

\subsection{Updates}

\begin{frame}{Updates}
\begin{itemize}
\item Found correct way to create workspace for combine with additional POIs\\
\item Tested multiple combinations of different additional POIs with different process scalings
\rar What is the impact on the postfit signal strength?
\end{itemize}

\end{frame}

\section{Results for Scaling (\ttbar + \bbbar)$\cdot$\num[round-precision=1]{1.2}}

\begin{frame}
\centering
\begin{huge}
\textbf{Results for Scaling (\ttbar + \bbbar)$\cdot$\num[round-precision=1]{1.2}}
\end{huge}

\end{frame}

\subsection{Nomenclature}
\begin{frame}{Nomenclature}
\begin{itemize}
\item $\mu$: Signal strength
\item r: Signal strength parameter in fit
\item r-Fit: Fit r as only POI
\item MDF: Fit with additional POI(s)\\
\rar following example with additional parameters r\_ttbb and r\_ttcc

\end{itemize}
\end{frame}

\subsection{Correlations}
\begin{frame}{Expectations via Fit to Asimov Data Set - Correlation Plots}

\begin{minipage}{0.48\textwidth}
\begin{center}
\includegraphics[width=\textwidth]{\pathToPlots/JTBDT_Spring17v10/wo_NP/asimov/r_ttbb_r_ttcc/63445464_ttHbb_N1000_r_ttbb_r_ttcc_ttbarPlusBBbar_1p2_correlationPlot_PostfitS_nominal_S_1p0.pdf}\\
\vskip -0.3cm
S+B r-Fit
\end{center}
\end{minipage}
\hfill
\begin{minipage}{0.48\textwidth}
\begin{center}
\includegraphics[width=\textwidth]{\pathToPlots/JTBDT_Spring17v10/wo_NP/asimov/r_ttbb_r_ttcc/63445464_ttHbb_N1000_r_ttbb_r_ttcc_ttbarPlusBBbar_1p2_correlationPlot_PostfitS_MDFnominal_S_1p0.pdf}\\
\vskip -0.3cm
S+B MDF
\end{center}
\end{minipage}
\begin{itemize}
\item new POIs substitute rate uncertainties for \ttbar + \bbbar and \ttbar + \ccbar processes, respectively
\item correlations change slightly
\end{itemize}
\end{frame}

\begin{frame}{Fit to PseudoData - Correlation Plots}
\begin{minipage}{\textwidth}

\begin{minipage}{0.48\textwidth}
\begin{center}
\includegraphics[width=\textwidth]{\pathToPlots/JTBDT_Spring17v10/wo_NP/PseudoData/r_ttbb_r_ttcc/63445464_ttHbb_N1000_r_ttbb_r_ttcc_ttbarPlusBBbar_1p2_correlationPlot_PostfitS_nominal_S_1p0.pdf}\\
\vskip -0.2cm
r-Fit
\end{center}
\end{minipage}
%\hfill
\begin{minipage}{0.48\textwidth}
\begin{center}
\includegraphics[width=\textwidth]{\pathToPlots/JTBDT_Spring17v10/wo_NP/PseudoData/r_ttbb_r_ttcc/63445464_ttHbb_N1000_r_ttbb_r_ttcc_ttbarPlusBBbar_1p2_correlationPlot_PostfitS_MDFnominal_S_1p0.pdf}\\
\vskip -0.2cm
MDF
\end{center}
\end{minipage}
\\

%
%\begin{minipage}{0.48\textwidth}
%\begin{center}
%\includegraphics[width=\textwidth]{\pathToPlots/JTBDT_Spring17v10/wo_NP/asimov/r_ttbb_r_ttcc/63445464_ttHbb_N1000_r_ttbb_r_ttcc_ttbarPlusBBbar_1p2_correlationPlot_PostfitS_nominal_S_1p0.pdf}\\
%\vskip -0.2cm
%Asimov r-Fit
%\end{center}
%\end{minipage}
%%\hfill
%\begin{minipage}{0.48\textwidth}
%\begin{center}
%\includegraphics[width=\textwidth]{\pathToPlots/JTBDT_Spring17v10/wo_NP/asimov/r_ttbb_r_ttcc/63445464_ttHbb_N1000_r_ttbb_r_ttcc_ttbarPlusBBbar_1p2_correlationPlot_PostfitS_MDFnominal_S_1p0.pdf}\\
%\vskip -0.2cm
%Asimov MDF
%\end{center}
%\end{minipage}
%
\end{minipage}
\begin{minipage}{\textwidth}

\begin{itemize}
\item r and r\_ttbb \SI[round-precision=0]{60}{\percent} anti-correlated
\item r and r\_ttcc \SI[round-precision=0]{35}{\percent} correlated
\item correlations slightly change, same trend as asimov expectation
\end{itemize}
\end{minipage}
\end{frame}

\subsection{Signal Strength}

\begin{frame}{Expectations via Fit to Asimov Data Set - Fit Results for Signal Strength}
\begin{table}
\centering
%\caption{Signal Strength Parameters for 63445464\_ttHbb\_N1000\_r\_ttbb\_r\_ttcc\_ttbarPlusBBbar\_1.2}
\begin{tabular}{cc}
\toprule
Pseudo Experiment & Mean $\pm$ Mean Error $\pm$ RMS $\pm$ Fitted Error\\
\midrule
r-Fit, $\mu=\num[round-precision=1]{1.0}$ & \num[color=KIT-Lila]{1.06967} $\pm$ \num[color=KIT-Lila]{0.0262016} $\pm$ \num{0.824413} $\pm$ \num{0.814887}\\
MDF-Fit, $\mu=\num[round-precision=1]{1.0}$ & \num[color=KIT-Orange]{1.00474} $\pm$ \num[color=KIT-Orange]{0.0267746} $\pm$ \num{0.842443} $\pm$ \num{0.835631}\\
r-Fit, $\mu=\num[round-precision=1]{0.0}$ & \num[color=KIT-Lila]{0.0703779} $\pm$ \num[color=KIT-Lila]{0.0259101} $\pm$ \num{0.803214} $\pm$ \num{0.77681}\\
MDF-Fit, $\mu=\num[round-precision=1]{0.0}$ & \num[color=KIT-Orange]{0.00601733} $\pm$ \num[color=KIT-Orange]{0.0261621} $\pm$ \num{0.811024} $\pm$ \num{0.796247}\\
\bottomrule
\end{tabular}
\end{table}
\begin{itemize}
\item bias visible for nominal case\\
\rar completely gone in MDF case
\item precision marginally worse in MDF case
\end{itemize}
\end{frame}


\begin{frame}{Fit to PseudoData - Fit Results for Signal Strength}
\begin{table}
\centering
%\caption{Signal Strength Parameters for 63445464\_ttHbb\_N1000\_r\_ttbb\_r\_ttcc\_ttbarPlusBBbar\_1.2}
\begin{tabular}{cc}
\toprule
Pseudo Experiment & Mean $\pm$ Mean Error $\pm$ RMS $\pm$ Fitted Error\\
\midrule
r-Fit, $\mu=\num[round-precision=1]{1.0}$ & \num[color=KIT-Lila]{1.06967} $\pm$ \num[color=KIT-Lila]{0.0262016} $\pm$ \num{0.824413} $\pm$ \num{0.814887}\\
MDF-Fit, $\mu=\num[round-precision=1]{1.0}$ & \num[color=KIT-Orange]{1.00474} $\pm$ \num[color=KIT-Orange]{0.0267746} $\pm$ \num{0.842443} $\pm$ \num{0.835631}\\
r-Fit, $\mu=\num[round-precision=1]{0.0}$ & \num[color=KIT-Lila]{0.0703779} $\pm$ \num[color=KIT-Lila]{0.0259101} $\pm$ \num{0.803214} $\pm$ \num{0.77681}\\
MDF-Fit, $\mu=\num[round-precision=1]{0.0}$ & \num[color=KIT-Orange]{0.00601733} $\pm$ \num[color=KIT-Orange]{0.0261621} $\pm$ \num{0.811024} $\pm$ \num{0.796247}\\
\bottomrule
\end{tabular}
\end{table}
\begin{itemize}
\item values compatible with asimov expectation
\item injected $\mu$ not inside range fitted \textcolor{KIT-Lila}{mean value $\pm$ mean error in r-Fit}\\
\rar \textcolor{KIT-Orange}{recovered in MDF case}
\item precision marginally worse in MDF case
\end{itemize}
\end{frame}


\begin{frame}{Expectations via Fit to Asimov Data Set - Pull Plots, $\mu = 0$}

\begin{minipage}{0.44\textwidth}
\centering
\includegraphics[width=\textwidth]{\pathToPlots/JTBDT_Spring17v10/wo_NP/asimov/r_ttbb_r_ttcc/63445464_ttHbb_N1000_r_ttbb_r_ttcc_ttbarPlusBBbar_1p2_pullplot_nominal_S_0p0.pdf}\\
\vskip -0.2cm
Pull Plot of r-Fit for $\mu = 0$

\end{minipage}
\hfill
\begin{minipage}{0.44\textwidth}
\centering
\includegraphics[width=\textwidth]{\pathToPlots/JTBDT_Spring17v10/wo_NP/asimov/r_ttbb_r_ttcc/63445464_ttHbb_N1000_r_ttbb_r_ttcc_ttbarPlusBBbar_1p2_pullplot_MDFnominal_S_0p0.pdf}\\
\vskip -0.2cm
Pull Plot of MDF for $\mu = 0$

\end{minipage}
\begin{itemize}
\item only parameter for \ttbar + \bbbar process is pulled
\item other parameters in MDF case stay on prefit value
\end{itemize}
\end{frame}

\subsection{Uncertainties}
\begin{frame}{Expectations via Fit to Asimov Data Set - Pull Plots, $\mu = 1$}

\begin{minipage}{0.44\textwidth}
\centering
\includegraphics[width=\textwidth]{\pathToPlots/JTBDT_Spring17v10/wo_NP/asimov/r_ttbb_r_ttcc/63445464_ttHbb_N1000_r_ttbb_r_ttcc_ttbarPlusBBbar_1p2_pullplot_nominal_S_1p0.pdf}\\
\vskip -0.2cm
Pull Plot of r-Fit for $\mu = 1$

\end{minipage}
\hfill
\begin{minipage}{0.44\textwidth}
\centering
\includegraphics[width=\textwidth]{\pathToPlots/JTBDT_Spring17v10/wo_NP/asimov/r_ttbb_r_ttcc/63445464_ttHbb_N1000_r_ttbb_r_ttcc_ttbarPlusBBbar_1p2_pullplot_MDFnominal_S_1p0.pdf}\\
\vskip -0.2cm
Pull Plot of MDF for $\mu = 1$

\end{minipage}
\begin{itemize}
\item B-Only fit shows pulls
\item S+B fit shows same behavior as fit for $\mu = 0$
\end{itemize}
\end{frame}


\begin{frame}{Fit to PseudoData - Pull Plots for $\mu = 0$}

\begin{minipage}{0.44\textwidth}
\begin{center}
\includegraphics[width=\textwidth]{\pathToPlots/JTBDT_Spring17v10/wo_NP/PseudoData/r_ttbb_r_ttcc/63445464_ttHbb_N1000_r_ttbb_r_ttcc_ttbarPlusBBbar_1p2_pullplot_nominal_S_0p0.pdf}\\
%\vskip -0.2cm
Pull Plot of r-Fit for $\mu = 0$
\end{center}

\end{minipage}
\hfill
\begin{minipage}{0.44\textwidth}
\begin{center}
\includegraphics[width=\textwidth]{\pathToPlots/JTBDT_Spring17v10/wo_NP/PseudoData/r_ttbb_r_ttcc/63445464_ttHbb_N1000_r_ttbb_r_ttcc_ttbarPlusBBbar_1p2_pullplot_MDFnominal_S_0p0.pdf}\\
%\vskip -0.2cm
Pull Plot of MDF for $\mu = 0$
\end{center}
\end{minipage}
\begin{itemize}
\item visible pulls for parameters other than \texttt{bgnorm\_ttbarPlusBBbar} (r-Fit) and r\_ttbb (MDF)\\
\rar expectation: normalizations of postfit process templates is different than expected values
\end{itemize}
\end{frame}

\begin{frame}{Fit to PseudoData - Pull Plots for $\mu = 1$}

\begin{minipage}{0.44\textwidth}
\centering
\includegraphics[width=\textwidth]{\pathToPlots/JTBDT_Spring17v10/wo_NP/PseudoData/r_ttbb_r_ttcc/63445464_ttHbb_N1000_r_ttbb_r_ttcc_ttbarPlusBBbar_1p2_pullplot_nominal_S_1p0.pdf}\\
%\vskip -0.2cm
Pull Plot of r-Fit for $\mu = 1$

\end{minipage}
\hfill
\begin{minipage}{0.44\textwidth}
\centering
\includegraphics[width=\textwidth]{\pathToPlots/JTBDT_Spring17v10/wo_NP/PseudoData/r_ttbb_r_ttcc/63445464_ttHbb_N1000_r_ttbb_r_ttcc_ttbarPlusBBbar_1p2_pullplot_MDFnominal_S_1p0.pdf}\\
%\vskip -0.2cm
Pull Plot of MDF for $\mu = 1$

\end{minipage}
\begin{itemize}
\item pulls in S+B fit less pronounced than in B-Only fit
\item other than that, same behavior as in $\mu = 0$ case
\end{itemize}
\end{frame}

\subsection{Normalizations for j $\geq 6$, t $\geq 4$}

\begin{frame}{Fit to PseudoData - Normalization Pull Plots for j$\geq 6$, t$\geq 4$, $\mu = 0$}

\begin{minipage}{0.44\textwidth}
\centering
\includegraphics[width=\textwidth]{\pathToPlots/JTBDT_Spring17v10/wo_NP/PseudoData/r_ttbb_r_ttcc/63445464_ttHbb_N1000_r_ttbb_r_ttcc_ttbarPlusBBbar_1p2_ljets_jge6_tge4_normalisation_pullplot_nominal_S_0p0.pdf}\\
%\vskip -0.2cm
Pull Plot of r-Fit $\mu = 0$

\end{minipage}
\hfill
\begin{minipage}{0.44\textwidth}
\centering
\includegraphics[width=\textwidth]{\pathToPlots/JTBDT_Spring17v10/wo_NP/PseudoData/r_ttbb_r_ttcc/63445464_ttHbb_N1000_r_ttbb_r_ttcc_ttbarPlusBBbar_1p2_ljets_jge6_tge4_normalisation_pullplot_MDFnominal_S_0p0.pdf}\\
%\vskip -0.2cm
Pull Plot of MDF for $\mu = 0$

\end{minipage}
\begin{itemize}
\item normalizations of signal processes fits expectation
\item background processes are overestimated
\end{itemize}
\end{frame}

\begin{frame}{Fit to PseudoData - Normalization Pull Plots for $\mu = 1$}

\begin{minipage}{0.44\textwidth}
\centering
\includegraphics[width=\textwidth]{\pathToPlots/JTBDT_Spring17v10/wo_NP/PseudoData/r_ttbb_r_ttcc/63445464_ttHbb_N1000_r_ttbb_r_ttcc_ttbarPlusBBbar_1p2_ljets_jge6_tge4_normalisation_pullplot_nominal_S_1p0.pdf}\\
%\vskip -0.2cm
Pull Plot of r-Fit for $\mu = 1$

\end{minipage}
\hfill
\begin{minipage}{0.44\textwidth}
\centering
\includegraphics[width=\textwidth]{\pathToPlots/JTBDT_Spring17v10/wo_NP/PseudoData/r_ttbb_r_ttcc/63445464_ttHbb_N1000_r_ttbb_r_ttcc_ttbarPlusBBbar_1p2_ljets_jge6_tge4_normalisation_pullplot_MDFnominal_S_1p0.pdf}\\
%\vskip -0.2cm
Pull Plot of MDF for $\mu = 1$

\end{minipage}
\begin{itemize}
\item same behavior as in $\mu = 0$ case
\end{itemize}
\end{frame}

%\sisetup{round-mode=places,round-precision=2}
\section{Current Status}
\subsection{Own PseudoData Generation}
\begin{frame}{Own PseudoData Generation}
\begin{block}{New Developments}
\begin{itemize}
\item Expert recommended new \texttt{combine} command
\item Result: still Bias visible for $\mu = 0$, not yet understood
\end{itemize}
\end{block}

\vskip -0.6cm
\begin{figure}
\centering
\subfloat[Distribution of $\mu$]{\includegraphics[width=0.4\textwidth]{plots/ttHbb_ttbb_N100_freezeAllNuisances_POI.pdf}}
\subfloat[Mean and Median Values]{\includegraphics[width=0.4\textwidth]{plots/ttHbb_ttbb_N100_freezeAllNuisances_POImeans.pdf}}
\caption[Fit Results for Example with Own PseudoData and additional \texttt{combine} command]{Fit Results for Example with Own PseudoData and additional \texttt{combine} command \texttt{--freezeNuisances all}.}
\end{figure}

\end{frame}
\subsection{Toy Generation with \texttt{combine}}
\begin{frame}{Toy Generation with \texttt{combine}}
\vskip -0.7cm
\begin{figure}
\centering
\subfloat[Distribution of $\mu = 0$]{\includegraphics[width=0.5\textwidth]{plots/ttHbb_ttbb_N100_allInOne_sig0_combineToys_freezeNuisancesAll_POI.pdf}}
\subfloat[Distribution of $\mu = 1$]{\includegraphics[width=0.45\textwidth]{plots/ttHbb_ttbb_N100_allInOne_sig1_combineToys_freezeNuisancesAll_POI.pdf}}
\caption[Fit Results for Example with \texttt{combine}-generated Toys]{Fit Results for Example with \texttt{combine}-generated Toys. Command: \texttt{combine -v 99 -M MaxLikelihoodFit -m 125 --freezeNuisances all --minimizerStrategy 0 --minimizerTolerance 0.001 --saveNormalizations --saveShapes --rMin=-5.00 --rMax=5.00 -t 100 --minos all --expectSignal \$signalStrength \$targetDatacard}.}
\end{figure}
\vskip -0.6cm
\begin{block}{Test with Build-In Toy Generation}
\begin{itemize}
\item generated 100 toys at once
\item results looked promising, tried to adapt the generation to existing scripts
\end{itemize}
\end{block}

\end{frame}
\subsection{Adapted Toy Generation}
\begin{frame}{Adapted Toy Generation Parameters}
\begin{block}{Idea}
\begin{itemize}
\item generate 100 Toys separately $\rightarrow$ recreate data structure of own PseudoData generation
\item important: always use different random seed (here: number between 0 and 99)
\item \texttt{combine} command: \texttt{combine -v 99 -M MaxLikelihoodFit -m 125 --freezeNuisances all --minimizerStrategy 0 --minimizerTolerance 0.001 --saveNormalizations --saveShapes --rMin=-5.00 --rMax=5.00 -t 1 --minos all -s \$randomSeed --expectSignal \$signalStrength \$targetDatacard} + additional options
\end{itemize}
\end{block}
\end{frame}

\begin{frame}{Adapted Toy Generation, no additional options}
\begin{figure}
\centering
\subfloat[Distribution of $\mu$]{\includegraphics[width=0.5\textwidth]{plots/ttHbb_ttbb_N100_combineToys_POI.pdf}}
\subfloat[Mean and Median Values]{\includegraphics[width=0.5\textwidth]{plots/ttHbb_ttbb_N100_combineToys_POImeans.pdf}}
\caption[Adapted Toy Generation without Additional Options]{Adapted Toy Generation without Additional Options}
\end{figure}
\end{frame}

\begin{frame}{Adapted Toy Generation with \texttt{--toysNoSystematics}}
\begin{figure}
\centering
\subfloat[Distribution of $\mu$]{\includegraphics[width=0.45\textwidth]{plots/ttHbb_ttbb_N100_combineToys_toysNoSystematics_POI.pdf}}
\subfloat[Mean and Median Values]{\includegraphics[width=0.45\textwidth]{plots/ttHbb_ttbb_N100_combineToys_toysNoSystematics_POImeans.pdf}}
\caption[Adapted Toy Generation with option \texttt{--toysNoSystematics}]{Adapted Toy Generation with option \texttt{--toysNoSystematics}}
\end{figure}
\end{frame}

\begin{frame}{Adapted Toy Generation with \texttt{--freezeNuisances all}}
\begin{figure}
\centering
\subfloat[Distribution of $\mu$]{\includegraphics[width=0.45\textwidth]{plots/ttHbb_ttbb_N100_combineToys_freezeNuisancesAll_POI.pdf}}
\subfloat[Mean and Median Values]{\includegraphics[width=0.45\textwidth]{plots/ttHbb_ttbb_N100_combineToys_freezeNuisancesAll_POImeans.pdf}}
\caption[Adapted Toy Generation with option \texttt{--freezeNuisances all}]{Adapted Toy Generation with option \texttt{--freezeNuisances all}}
\end{figure}
\end{frame}

\begin{frame}{Adapted Toy Generation with \texttt{--toysNoSystematics --freezeNuisances all}}
\begin{figure}
\centering
\subfloat[Distribution of $\mu$]{\includegraphics[width=0.45\textwidth]{plots/ttHbb_ttbb_N100_combineToys_toysNoSystematics_freezeNuisancesAll_POI.pdf}}
\subfloat[Mean and Median Values]{\includegraphics[width=0.45\textwidth]{plots/ttHbb_ttbb_N100_combineToys_toysNoSystematics_freezeNuisancesAll_POImeans.pdf}}
\caption[Adapted Toy Generation with options \texttt{--toysNoSystematics --freezeNuisances all}]{Adapted Toy Generation with options \texttt{--toysNoSystematics --freezeNuisances all}}
\end{figure}
\end{frame}

\section{Summary}
\begin{frame}{Summary}
\begin{itemize}
\item Conducted test for different JES scalings\\
\rar Constraints do change, stronger for larger a priori uncertainties

\item artificial equalization of one sigma region via datacard factors works for limited range of extrapolation
\end{itemize}

\end{frame}



%%% BACKUP

\beginbackup

\section*{Backup}
\begin{frame}
	\begin{center}
		\textbf{\Huge Backup}
	\end{center}
\end{frame}
%\subsection*{Datacard}

%\subsection*{List of Processes}\label{backup::listOfProcesses}
\begin{frame}{List of Processes}

\lstinputlisting[numbers=none, basicstyle=\tiny,  frame=none]{../listOfProcesses.txt}
\end{frame}

%\subsection*{List of Nuisance Parameters}\label{backup::listOfNP}

\begin{frame}{\texttt{combine} Commands - 1}
\begin{block}{For Toy Generation}
\texttt{combine -M GenerateOnly -m 125 --saveToys -t \$numberOfToysPerExperiment -n \$suffix --expectSignal \$signalStrength -s \$randomseed --toysFrequentist \$toyDatacard}
\end{block}
\begin{block}{For MaxLikelihood Fit}
\texttt{combine -M MaxLikelihoodFit -m 125 --minimizerStrategy 0 --minimizerTolerance 0.001 --saveNormalizations --saveShapes --rMin=-10.00 --rMax=10.00 -t \$numberOfToysPerExperiment --toysFile \$toyFile --minos all \$targetDatacard}
\end{block}

\end{frame}

\begin{frame}{\texttt{combine} Commands - 2}
\begin{block}{Create workspace with multiple POIs}
\texttt{text2workspace.py -P HiggsAnalysis.CombinedLimit.PhysicsModel:multiSignalModel --PO verbose  --PO 'map=.$^*$/(ttH\_$^*$):r[1,-10,10]' --PO 'map=.$^*$/(\$processName):\$POIname[1,-10,10]' \$targetDatacard -o \$workspaceName}
\end{block}
\begin{itemize}
\item use combine command from previous slide, where \$targetDatacard is replaced by \$workspaceName
\end{itemize}
\end{frame}

\newcommand{\createResultFrames}[2]{
%#1: additional POI in path form
%#2: additional POI in text form



\begin{frame}{Links for POI #2 - 1}
\begin{minipage}[l]{0.48\textwidth}

\begin{itemize}
\item \hyperlink{#1_ttbarOther_0p9}{\color{blue}(\ttbar + \ccbar)$\cdot$\num[round-precision=1]{0.9}}
\item \hyperlink{#1_ttbarOther_0p99}{\color{blue}(\ttbar + \ccbar)$\cdot$\num[round-precision=1]{0.99}}
\item \hyperlink{#1_ttbarOther_1p01}{\color{blue}(\ttbar + \ccbar)$\cdot$\num[round-precision=1]{1.01}}
\item \hyperlink{#1_ttbarOther_1p1}{\color{blue}(\ttbar + \ccbar)$\cdot$\num[round-precision=1]{1.1}}

\item \hyperlink{#1_ttbarPlusCCbar_0p5}{\color{blue}(\ttbar + \ccbar)$\cdot$\num[round-precision=1]{0.5}}
\item \hyperlink{#1_ttbarPlusCCbar_0p8}{\color{blue}(\ttbar + \ccbar)$\cdot$\num[round-precision=1]{0.8}}
\item \hyperlink{#1_ttbarPlusCCbar_1p2}{\color{blue}(\ttbar + \ccbar)$\cdot$\num[round-precision=1]{1.2}}
\item \hyperlink{#1_ttbarPlusCCbar_1p5}{\color{blue}(\ttbar + \ccbar)$\cdot$\num[round-precision=1]{1.5}}


\item \hyperlink{#1_ttbarPlusBBbar_0p5}{\color{blue}(\ttbar + \bbbar)$\cdot$\num[round-precision=1]{0.5}}
\item \hyperlink{#1_ttbarPlusBBbar_0p8}{\color{blue}(\ttbar + \bbbar)$\cdot$\num[round-precision=1]{0.8}}
\item \hyperlink{#1_ttbarPlusBBbar_1p2}{\color{blue}(\ttbar + \bbbar)$\cdot$\num[round-precision=1]{1.2}}
\item \hyperlink{#1_ttbarPlusBBbar_1p5}{\color{blue}(\ttbar + \bbbar)$\cdot$\num[round-precision=1]{1.5}}

\end{itemize}
\end{minipage}
\hfill
\begin{minipage}[r]{0.48\textwidth}

\begin{itemize}
\item \hyperlink{#1_ttbarPlusB_0p5}{\color{blue}(\ttbar + B)$\cdot$\num[round-precision=1]{0.5}}
\item \hyperlink{#1_ttbarPlusB_0p8}{\color{blue}(\ttbar + B)$\cdot$\num[round-precision=1]{0.8}}
\item \hyperlink{#1_ttbarPlusB_1p2}{\color{blue}(\ttbar + B)$\cdot$\num[round-precision=1]{1.2}}
\item \hyperlink{#1_ttbarPlusB_1p5}{\color{blue}(\ttbar + B)$\cdot$\num[round-precision=1]{1.5}}

\item \hyperlink{#1_ttbarPlus2B_0p5}{\color{blue}(\ttbar + 2B)$\cdot$\num[round-precision=1]{0.5}}
\item \hyperlink{#1_ttbarPlus2B_0p8}{\color{blue}(\ttbar + 2B)$\cdot$\num[round-precision=1]{0.8}}
\item \hyperlink{#1_ttbarPlus2B_1p2}{\color{blue}(\ttbar + 2B)$\cdot$\num[round-precision=1]{1.2}}
\item \hyperlink{#1_ttbarPlus2B_1p5}{\color{blue}(\ttbar + 2B)$\cdot$\num[round-precision=1]{1.5}}


\item \hyperlink{#1_ttbarPlusBBbar_ttbarPlusB_0p5_0p5}{\color{blue}[(\ttbar + \bbbar), (\ttbar + B)]$\cdot$\num[round-precision=1]{0.5}}
\item \hyperlink{#1_ttbarPlusBBbar_ttbarPlusB_0p8_0p8}{\color{blue}[(\ttbar + \bbbar), (\ttbar + B)]$\cdot$\num[round-precision=1]{0.8}}
\item \hyperlink{#1_ttbarPlusBBbar_ttbarPlusB_1p2_1p2}{\color{blue}[(\ttbar + \bbbar), (\ttbar + B)]$\cdot$\num[round-precision=1]{1.2}}
\item \hyperlink{#1_ttbarPlusBBbar_ttbarPlusB_1p5_1p5}{\color{blue}[(\ttbar + \bbbar), (\ttbar + B)]$\cdot$\num[round-precision=1]{1.5}}

\end{itemize}
\end{minipage}
\end{frame}

\begin{frame}{Links for POI #2 - 2}

\begin{itemize}
\item \hyperlink{#1_ttbarPlusBBbar_ttbarPlus2B_0p5_0p5}{\color{blue}[(\ttbar + \bbbar), (\ttbar + 2B)]$\cdot$\num[round-precision=1]{0.5}}
\item \hyperlink{#1_ttbarPlusBBbar_ttbarPlus2B_0p8_0p8}{\color{blue}[(\ttbar + \bbbar), (\ttbar + 2B)]$\cdot$\num[round-precision=1]{0.8}}
\item \hyperlink{#1_ttbarPlusBBbar_ttbarPlus2B_1p2_1p2}{\color{blue}[(\ttbar + \bbbar), (\ttbar + 2B)]$\cdot$\num[round-precision=1]{1.2}}
\item \hyperlink{#1_ttbarPlusBBbar_ttbarPlus2B_1p5_1p5}{\color{blue}[(\ttbar + \bbbar), (\ttbar + 2B)]$\cdot$\num[round-precision=1]{1.5}}

\item \hyperlink{#1_ttbarPlus2B_ttbarPlusB_0p5_0p5}{\color{blue}[(\ttbar + 2B), (\ttbar + B)]$\cdot$\num[round-precision=1]{0.5}}
\item \hyperlink{#1_ttbarPlus2B_ttbarPlusB_0p8_0p8}{\color{blue}[(\ttbar + 2B), (\ttbar + B)]$\cdot$\num[round-precision=1]{0.8}}
\item \hyperlink{#1_ttbarPlus2B_ttbarPlusB_1p2_1p2}{\color{blue}[(\ttbar + 2B), (\ttbar + B)]$\cdot$\num[round-precision=1]{1.2}}
\item \hyperlink{#1_ttbarPlus2B_ttbarPlusB_1p5_1p5}{\color{blue}[(\ttbar + 2B), (\ttbar + B)]$\cdot$\num[round-precision=1]{1.5}}


\item \hyperlink{#1_ttbarPlusBBbar_ttbarPlus2B_ttbarPlusB_0p5_0p5_0p5}{\color{blue}[(\ttbar + \bbbar), (\ttbar + 2B), (\ttbar + B)]$\cdot$\num[round-precision=1]{0.5}}
\item \hyperlink{#1_ttbarPlusBBbar_ttbarPlus2B_ttbarPlusB_0p8_0p8_0p8}{\color{blue}[(\ttbar + \bbbar), (\ttbar + 2B), (\ttbar + B)]$\cdot$\num[round-precision=1]{0.8}}
\item \hyperlink{#1_ttbarPlusBBbar_ttbarPlus2B_ttbarPlusB_1p2_1p2_1p2}{\color{blue}[(\ttbar + \bbbar), (\ttbar + 2B), (\ttbar + B)]$\cdot$\num[round-precision=1]{1.2}}
\item \hyperlink{#1_ttbarPlusBBbar_ttbarPlus2B_ttbarPlusB_1p5_1p5_1p5}{\color{blue}[(\ttbar + \bbbar), (\ttbar + 2B), (\ttbar + B)]$\cdot$\num[round-precision=1]{1.5}}

\end{itemize}


\end{frame}
\compareToAsimov{#1}{#2}{ttbarOther_0p9}{(\ttbar + lf)$\cdot$ \num[round-precision=1]{0.9}}
\compareToAsimov{#1}{#2}{ttbarOther_0p99}{(\ttbar + lf)$\cdot$ \num[round-precision=2]{0.99}}
\compareToAsimov{#1}{#2}{ttbarOther_1p01}{(\ttbar + lf)$\cdot$ \num[round-precision=2]{1.01}}
\compareToAsimov{#1}{#2}{ttbarOther_1p1}{(\ttbar + lf)$\cdot$ \num[round-precision=1]{1.1}}


\compareToAsimov{#1}{#2}{ttbarPlusCCbar_0p5}{(\ttbar + \ccbar)$\cdot$ \num[round-precision=1]{0.5}}
\compareToAsimov{#1}{#2}{ttbarPlusCCbar_0p8}{(\ttbar + \ccbar)$\cdot$ \num[round-precision=1]{0.8}}
\compareToAsimov{#1}{#2}{ttbarPlusCCbar_1p2}{(\ttbar + \ccbar)$\cdot$ \num[round-precision=1]{1.2}}
\compareToAsimov{#1}{#2}{ttbarPlusCCbar_1p5}{(\ttbar + \ccbar)$\cdot$ \num[round-precision=1]{1.5}}

\compareToAsimov{#1}{#2}{ttbarPlusBBbar_0p5}{(\ttbar + \bbbar)$\cdot$ \num[round-precision=1]{0.5}}
\compareToAsimov{#1}{#2}{ttbarPlusBBbar_0p8}{(\ttbar + \bbbar)$\cdot$ \num[round-precision=1]{0.8}}
\compareToAsimov{#1}{#2}{ttbarPlusBBbar_1p2}{(\ttbar + \bbbar)$\cdot$ \num[round-precision=1]{1.2}}
\compareToAsimov{#1}{#2}{ttbarPlusBBbar_1p5}{(\ttbar + \bbbar)$\cdot$ \num[round-precision=1]{1.5}}

\compareToAsimov{#1}{#2}{ttbarPlusBBbar_0p5}{(\ttbar + \bbbar)$\cdot$ \num[round-precision=1]{0.5}}
\compareToAsimov{#1}{#2}{ttbarPlusBBbar_0p8}{(\ttbar + \bbbar)$\cdot$ \num[round-precision=1]{0.8}}
\compareToAsimov{#1}{#2}{ttbarPlusBBbar_1p2}{(\ttbar + \bbbar)$\cdot$ \num[round-precision=1]{1.2}}
\compareToAsimov{#1}{#2}{ttbarPlusBBbar_1p5}{(\ttbar + \bbbar)$\cdot$ \num[round-precision=1]{1.5}}

\compareToAsimov{#1}{#2}{ttbarPlusB_0p5}{(\ttbar + B)$\cdot$ \num[round-precision=1]{0.5}}
\compareToAsimov{#1}{#2}{ttbarPlusB_0p8}{(\ttbar + B)$\cdot$ \num[round-precision=1]{0.8}}
\compareToAsimov{#1}{#2}{ttbarPlusB_1p2}{(\ttbar + B)$\cdot$ \num[round-precision=1]{1.2}}
\compareToAsimov{#1}{#2}{ttbarPlusB_1p5}{(\ttbar + B)$\cdot$ \num[round-precision=1]{1.5}}

\compareToAsimov{#1}{#2}{ttbarPlus2B_0p5}{(\ttbar + 2B)$\cdot$ \num[round-precision=1]{0.5}}
\compareToAsimov{#1}{#2}{ttbarPlus2B_0p8}{(\ttbar + 2B)$\cdot$ \num[round-precision=1]{0.8}}
\compareToAsimov{#1}{#2}{ttbarPlus2B_1p2}{(\ttbar + 2B)$\cdot$ \num[round-precision=1]{1.2}}
\compareToAsimov{#1}{#2}{ttbarPlus2B_1p5}{(\ttbar + 2B)$\cdot$ \num[round-precision=1]{1.5}}

\compareToAsimov{#1}{#2}{ttbarPlusBBbar_ttbarPlusB_0p5_0p5}{[(\ttbar + \bbbar), (\ttbar + B)]$\cdot$\num[round-precision=1]{0.5}}
\compareToAsimov{#1}{#2}{ttbarPlusBBbar_ttbarPlusB_0p8_0p8}{[(\ttbar + \bbbar), (\ttbar + B)]$\cdot$\num[round-precision=1]{0.8}}
\compareToAsimov{#1}{#2}{ttbarPlusBBbar_ttbarPlusB_1p2_1p2}{[(\ttbar + \bbbar), (\ttbar + B)]$\cdot$\num[round-precision=1]{1.2}}
\compareToAsimov{#1}{#2}{ttbarPlusBBbar_ttbarPlusB_1p5_1p5}{[(\ttbar + \bbbar), (\ttbar + B)]$\cdot$\num[round-precision=1]{1.5}}

\compareToAsimov{#1}{#2}{ttbarPlusBBbar_ttbarPlus2B_0p5_0p5}{[(\ttbar + \bbbar), (\ttbar + 2B)]$\cdot$\num[round-precision=1]{0.5}}
\compareToAsimov{#1}{#2}{ttbarPlusBBbar_ttbarPlus2B_0p8_0p8}{[(\ttbar + \bbbar), (\ttbar + 2B)]$\cdot$\num[round-precision=1]{0.8}}
\compareToAsimov{#1}{#2}{ttbarPlusBBbar_ttbarPlus2B_1p2_1p2}{[(\ttbar + \bbbar), (\ttbar + 2B)]$\cdot$\num[round-precision=1]{1.2}}
\compareToAsimov{#1}{#2}{ttbarPlusBBbar_ttbarPlus2B_1p5_1p5}{[(\ttbar + \bbbar), (\ttbar + 2B)]$\cdot$\num[round-precision=1]{1.5}}

\compareToAsimov{#1}{#2}{ttbarPlus2B_ttbarPlusB_0p5_0p5}{[(\ttbar + 2B), (\ttbar + B)]$\cdot$\num[round-precision=1]{0.5}}
\compareToAsimov{#1}{#2}{ttbarPlus2B_ttbarPlusB_0p8_0p8}{[(\ttbar + 2B), (\ttbar + B)]$\cdot$\num[round-precision=1]{0.8}}
\compareToAsimov{#1}{#2}{ttbarPlus2B_ttbarPlusB_1p2_1p2}{[(\ttbar + 2B), (\ttbar + B)]$\cdot$\num[round-precision=1]{1.2}}
\compareToAsimov{#1}{#2}{ttbarPlus2B_ttbarPlusB_1p5_1p5}{[(\ttbar + 2B), (\ttbar + B)]$\cdot$\num[round-precision=1]{1.5}}

\compareToAsimov{#1}{#2}{ttbarPlusBBbar_ttbarPlus2B_ttbarPlusB_0p5_0p5_0p5}{[(\ttbar + \bbbar), (\ttbar + 2B), (\ttbar + B)]$\cdot$\num[round-precision=1]{0.5}}
\compareToAsimov{#1}{#2}{ttbarPlusBBbar_ttbarPlus2B_ttbarPlusB_0p8_0p8_0p8}{[(\ttbar + \bbbar), (\ttbar + 2B), (\ttbar + B)]$\cdot$\num[round-precision=1]{0.8}}
\compareToAsimov{#1}{#2}{ttbarPlusBBbar_ttbarPlus2B_ttbarPlusB_1p2_1p2_1p2}{[(\ttbar + \bbbar), (\ttbar + 2B), (\ttbar + B)]$\cdot$\num[round-precision=1]{1.2}}
\compareToAsimov{#1}{#2}{ttbarPlusBBbar_ttbarPlus2B_ttbarPlusB_1p5_1p5_1p5}{[(\ttbar + \bbbar), (\ttbar + 2B), (\ttbar + B)]$\cdot$\num[round-precision=1]{1.5}}
}

%\createResultFrames{r_ttbb}{r\_ttbb}
% PUT BACKUP SLIDES HERE

%%%

\backupend
\end{document}





