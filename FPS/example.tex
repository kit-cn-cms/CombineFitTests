\sisetup{round-mode=places,round-precision=3, scientific-notation=fixed, fixed-exponent=0}
\section{Analysis Strategy}
\begin{frame}
\centering
\begin{huge}
\textbf{Analysis Strategy}
\end{huge}

\end{frame}


\subsection{Input for Analysis}

\begin{frame}{Input for Analysis}

\begin{block}{Input Parameters}
\begin{itemize}
\item datacard: Spring17v10 \href{https://indico.cern.ch/event/628833/contributions/2624507/attachments/1474840/2283855/KITv10p2.pdf}{\textcolor{blue}{go to presentation}}
\item default categories: $j\geq 6, t=3$; $j=4, t=4$; $j=5, t\geq 4$; $j\geq 6, t\geq 4$ (63-44-54-64)
\item signal process: \ttbarH(\bbbar)
\item background processes: \ttbar+lf, \ttbar+b, \ttbar+2b, \ttbar+\bbbar, \ttbar + \ccbar
%\item only considering b-tagging, cross section, lumi and lepton efficiency uncertainties, no JES uncertainties\\
\item list of considered uncertainties can be found on the next slide
\end{itemize}
\end{block}
\begin{itemize}
\item Following example shows results for combined fit in default categories with $\num[round-precision=1]{1.2}\cdot$(\ttbar+\bbbar)
\end{itemize}

\end{frame}

\begin{frame}{List of Nuisance Parameters}

\lstinputlisting[numbers=none, basicstyle=\tiny,  frame=none]{../listOfNP.txt}
\end{frame}

\subsection{Analysis Steps}

\subsubsection{Input Histograms}
\begin{frame}{Templates for Category 64}

\begin{minipage}{0.48\textwidth}
\begin{center}
\includegraphics[width=\textwidth]{\pathToPlots/JTBDT_Spring17v10/ljets_jge6_tge4_JTBDT_Spring17v10_original_stackplot.pdf}\label{fig::stackPlot64}\\
Stacked Plot of BDT Template Histograms

\end{center}
\end{minipage}
\begin{minipage}{0.48\textwidth}
\begin{center}
\includegraphics[width=\textwidth]{\pathToPlots/JTBDT_Spring17v10/ljets_jge6_tge4_JTBDT_Spring17v10_original_normed_stackplot.pdf}\label{fig::normed_stackPlot64}\\
Normalized BDT Template Histograms

\end{center}

\end{minipage}



\end{frame}
\subsubsection{Workflow}
\begin{frame}{Toy Generation}

%
%\subsubsection{Step 1: Scale Histograms}
%\begin{frame}{Step 1: Scale Histograms}
%
%\begin{minipage}{0.48\textwidth}
%\begin{center}
%\includegraphics[width=\textwidth]{\pathToPlots/JTBDT_Spring17v10/ljets_jge6_tge4_JTBDT_Spring17v10_original_stackplot.pdf}\\
%Stacked Plot of Original BDT Template Histograms
%\label{fig::compare_stackPlot64}
%\end{center}
%
%\end{minipage}
%\begin{minipage}{0.48\textwidth}
%\begin{center}
%\includegraphics[width=\textwidth]{\pathToPlots/JTBDT_Spring17v10/ljets_jge6_tge4_JTBDT_Spring17v10_ttHbb_scaled_ttbarPlusBBbar_1p2_stackplot.pdf}\\
%Stacked Plot of BDT Template Histograms with \ttbar+\bbbar Scaled by \num[round-precision=1]{1.2}
%\label{fig::compare_scaled_stackPlot64}
%\end{center}
%
%\end{minipage}
%
%
%
%
%\end{frame}
%
%\subsubsection{Step 2: Generate Toy Data}
%\begin{frame}{Step 2: Generate Toy Data}
%\begin{block}{Generate Toys from Scaled BDT Template}
%\begin{itemize}
%\item Sample toy measurement from Poisson distribution around bin mean
%\item Use different injected signal strengths for toy generation (nominal signal strength)
%\item Systematic uncertainties considered according to assumed priors when throwing the toys (via using \texttt{combine})
%\item Generate \num{1000} toys for each nominal signal strength
%\end{itemize}
%\end{block}
%\end{frame}
%
%\subsubsection{Step 3: Perform Fit}
%
%\begin{frame}{Step 3: Perform Fit}
%\begin{block}{Fit Original Histograms to Toys from Scaled Data}
%\begin{itemize}
%\item Fit original histograms to generated toy\\
%$\rightarrow$ Perform MaxLikelihoodFit with signal strength as POI via \texttt{combine}\\
%$\rightarrow$ Check distributions
%\end{itemize}
%\end{block}
\begin{block}{Step 1: Scale Process Template}
\begin{itemize}
\item scale specific process template(s) with constant factor\\
\rar create PseudoData corresponding to cross section that differs from expectation
\end{itemize}
\end{block}

\begin{block}{Step 2: Generate Toy Data}
\begin{itemize}
\item Sample toy measurement from Poisson distribution around bin mean
\item Use different injected signal strengths for toy generation (nominal signal strength)
\item Systematic uncertainties considered according to assumed priors when throwing the toys (via using \texttt{combine})
\end{itemize}
\end{block}

\end{frame}

\begin{frame}{Analyzing the Data}

\begin{block}{Step 3: Perform Fit}
\begin{itemize}
\item Fit original templates to generated toys\\
\rar Perform \texttt{MaxLikelihoodFit} with signal strength as POI\\
\rar Perform \texttt{MaxLikelihoodFit} with additional POIs for scaled processes\\
\rar Collect distributions
\end{itemize}
\end{block}

\begin{block}{Collect Fit Results}
\begin{itemize}
\item Analyze distribution of fit results for each parameter, respectively
\item Is the average POI equal to the injected signal strength during toy generation? 
\end{itemize}
\end{block}

\end{frame}

\subsection{Updates}

\begin{frame}{Updates}
\begin{itemize}
\item Found correct way to create workspace for combine with additional POIs\\
\item Tested multiple combinations of different additional POIs with different process scalings
\rar What is the impact on the postfit signal strength?
\end{itemize}

\end{frame}

\section{Results for Scaling (\ttbar + \bbbar)$\cdot$\num[round-precision=1]{1.2}}

\begin{frame}
\centering
\begin{huge}
\textbf{Results for Scaling (\ttbar + \bbbar)$\cdot$\num[round-precision=1]{1.2}}
\end{huge}

\end{frame}

\subsection{Nomenclature}
\begin{frame}{Nomenclature}
\begin{itemize}
\item $\mu$: Signal strength
\item r: Signal strength parameter in fit
\item r-Fit: Fit r as only POI
\item MDF: Fit with additional POI(s)\\
\rar following example with additional parameters r\_ttbb and r\_ttcc

\end{itemize}
\end{frame}

\subsection{Correlations}
\begin{frame}{Expectations via Fit to Asimov Data Set - Correlation Plots}

\begin{minipage}{0.48\textwidth}
\begin{center}
\includegraphics[width=\textwidth]{\pathToPlots/JTBDT_Spring17v10/wo_NP/asimov/r_ttbb_r_ttcc/63445464_ttHbb_N1000_r_ttbb_r_ttcc_ttbarPlusBBbar_1p2_correlationPlot_PostfitS_nominal_S_1p0.pdf}\\
\vskip -0.3cm
S+B r-Fit
\end{center}
\end{minipage}
\hfill
\begin{minipage}{0.48\textwidth}
\begin{center}
\includegraphics[width=\textwidth]{\pathToPlots/JTBDT_Spring17v10/wo_NP/asimov/r_ttbb_r_ttcc/63445464_ttHbb_N1000_r_ttbb_r_ttcc_ttbarPlusBBbar_1p2_correlationPlot_PostfitS_MDFnominal_S_1p0.pdf}\\
\vskip -0.3cm
S+B MDF
\end{center}
\end{minipage}
\begin{itemize}
\item new POIs substitute rate uncertainties for \ttbar + \bbbar and \ttbar + \ccbar processes, respectively
\item correlations change slightly
\end{itemize}
\end{frame}

\begin{frame}{Fit to PseudoData - Correlation Plots}
\begin{minipage}{\textwidth}

\begin{minipage}{0.48\textwidth}
\begin{center}
\includegraphics[width=\textwidth]{\pathToPlots/JTBDT_Spring17v10/wo_NP/PseudoData/r_ttbb_r_ttcc/63445464_ttHbb_N1000_r_ttbb_r_ttcc_ttbarPlusBBbar_1p2_correlationPlot_PostfitS_nominal_S_1p0.pdf}\\
\vskip -0.2cm
r-Fit
\end{center}
\end{minipage}
%\hfill
\begin{minipage}{0.48\textwidth}
\begin{center}
\includegraphics[width=\textwidth]{\pathToPlots/JTBDT_Spring17v10/wo_NP/PseudoData/r_ttbb_r_ttcc/63445464_ttHbb_N1000_r_ttbb_r_ttcc_ttbarPlusBBbar_1p2_correlationPlot_PostfitS_MDFnominal_S_1p0.pdf}\\
\vskip -0.2cm
MDF
\end{center}
\end{minipage}
\\

%
%\begin{minipage}{0.48\textwidth}
%\begin{center}
%\includegraphics[width=\textwidth]{\pathToPlots/JTBDT_Spring17v10/wo_NP/asimov/r_ttbb_r_ttcc/63445464_ttHbb_N1000_r_ttbb_r_ttcc_ttbarPlusBBbar_1p2_correlationPlot_PostfitS_nominal_S_1p0.pdf}\\
%\vskip -0.2cm
%Asimov r-Fit
%\end{center}
%\end{minipage}
%%\hfill
%\begin{minipage}{0.48\textwidth}
%\begin{center}
%\includegraphics[width=\textwidth]{\pathToPlots/JTBDT_Spring17v10/wo_NP/asimov/r_ttbb_r_ttcc/63445464_ttHbb_N1000_r_ttbb_r_ttcc_ttbarPlusBBbar_1p2_correlationPlot_PostfitS_MDFnominal_S_1p0.pdf}\\
%\vskip -0.2cm
%Asimov MDF
%\end{center}
%\end{minipage}
%
\end{minipage}
\begin{minipage}{\textwidth}

\begin{itemize}
\item r and r\_ttbb \SI[round-precision=0]{60}{\percent} anti-correlated
\item r and r\_ttcc \SI[round-precision=0]{35}{\percent} correlated
\item correlations slightly change, same trend as asimov expectation
\end{itemize}
\end{minipage}
\end{frame}

\subsection{Signal Strength}

\begin{frame}{Expectations via Fit to Asimov Data Set - Fit Results for Signal Strength}
\begin{table}
\centering
%\caption{Signal Strength Parameters for 63445464\_ttHbb\_N1000\_r\_ttbb\_r\_ttcc\_ttbarPlusBBbar\_1.2}
\begin{tabular}{cc}
\toprule
Pseudo Experiment & Mean $\pm$ Fitted Error\\
\midrule
r-Fit, $\mu=\num[round-precision=1]{1.0}$ & \num{1.05242} $\pm$ \num{0.796871}\\
MDF-Fit, $\mu=\num[round-precision=1]{1.0}$ & \num{1.00033} $\pm$ \num{0.812968}\\
r-Fit, $\mu=\num[round-precision=1]{0.0}$ & \num{0.0497557} $\pm$ \num{0.758759}\\
MDF-Fit, $\mu=\num[round-precision=1]{0.0}$ & \num{-0.000390852} $\pm$ \num{0.774408}\\
\bottomrule
\end{tabular}
\end{table}
\begin{itemize}
\item bias visible for nominal case\\
\rar completely gone in MDF case
\item precision marginally worse in MDF case
\end{itemize}
\end{frame}


\begin{frame}{Fit to PseudoData - Fit Results for Signal Strength}
\begin{table}
\centering
%\caption{Signal Strength Parameters for 63445464\_ttHbb\_N1000\_r\_ttbb\_r\_ttcc\_ttbarPlusBBbar\_1.2}
\begin{tabular}{cc}
\toprule
Pseudo Experiment & Mean $\pm$ Fitted Error\\
\midrule
r-Fit, $\mu=\num[round-precision=1]{1.0}$ & \num{1.05242} $\pm$ \num{0.796871}\\
MDF-Fit, $\mu=\num[round-precision=1]{1.0}$ & \num{1.00033} $\pm$ \num{0.812968}\\
r-Fit, $\mu=\num[round-precision=1]{0.0}$ & \num{0.0497557} $\pm$ \num{0.758759}\\
MDF-Fit, $\mu=\num[round-precision=1]{0.0}$ & \num{-0.000390852} $\pm$ \num{0.774408}\\
\bottomrule
\end{tabular}
\end{table}
\begin{itemize}
\item values compatible with asimov expectation
\item injected $\mu$ not inside range fitted \textcolor{KIT-Lila}{mean value $\pm$ mean error in r-Fit}\\
\rar \textcolor{KIT-Orange}{recovered in MDF case}
\item precision marginally worse in MDF case
\end{itemize}
\end{frame}


\begin{frame}{Expectations via Fit to Asimov Data Set - Pull Plots, $\mu = 0$}

\begin{minipage}{0.44\textwidth}
\centering
\includegraphics[width=\textwidth]{\pathToPlots/JTBDT_Spring17v10/wo_NP/asimov/r_ttbb_r_ttcc/63445464_ttHbb_N1000_r_ttbb_r_ttcc_ttbarPlusBBbar_1p2_pullplot_nominal_S_0p0.pdf}\\
\vskip -0.2cm
Pull Plot of r-Fit for $\mu = 0$

\end{minipage}
\hfill
\begin{minipage}{0.44\textwidth}
\centering
\includegraphics[width=\textwidth]{\pathToPlots/JTBDT_Spring17v10/wo_NP/asimov/r_ttbb_r_ttcc/63445464_ttHbb_N1000_r_ttbb_r_ttcc_ttbarPlusBBbar_1p2_pullplot_MDFnominal_S_0p0.pdf}\\
\vskip -0.2cm
Pull Plot of MDF for $\mu = 0$

\end{minipage}
\begin{itemize}
\item only parameter for \ttbar + \bbbar process is pulled
\item other parameters in MDF case stay on prefit value
\end{itemize}
\end{frame}

\subsection{Uncertainties}
\begin{frame}{Expectations via Fit to Asimov Data Set - Pull Plots, $\mu = 1$}

\begin{minipage}{0.44\textwidth}
\centering
\includegraphics[width=\textwidth]{\pathToPlots/JTBDT_Spring17v10/wo_NP/asimov/r_ttbb_r_ttcc/63445464_ttHbb_N1000_r_ttbb_r_ttcc_ttbarPlusBBbar_1p2_pullplot_nominal_S_1p0.pdf}\\
\vskip -0.2cm
Pull Plot of r-Fit for $\mu = 1$

\end{minipage}
\hfill
\begin{minipage}{0.44\textwidth}
\centering
\includegraphics[width=\textwidth]{\pathToPlots/JTBDT_Spring17v10/wo_NP/asimov/r_ttbb_r_ttcc/63445464_ttHbb_N1000_r_ttbb_r_ttcc_ttbarPlusBBbar_1p2_pullplot_MDFnominal_S_1p0.pdf}\\
\vskip -0.2cm
Pull Plot of MDF for $\mu = 1$

\end{minipage}
\begin{itemize}
\item B-Only fit shows pulls
\item S+B fit shows same behavior as fit for $\mu = 0$
\end{itemize}
\end{frame}


\begin{frame}{Fit to PseudoData - Pull Plots for $\mu = 0$}

\begin{minipage}{0.44\textwidth}
\begin{center}
\includegraphics[width=\textwidth]{\pathToPlots/JTBDT_Spring17v10/wo_NP/PseudoData/r_ttbb_r_ttcc/63445464_ttHbb_N1000_r_ttbb_r_ttcc_ttbarPlusBBbar_1p2_pullplot_nominal_S_0p0.pdf}\\
%\vskip -0.2cm
Pull Plot of r-Fit for $\mu = 0$
\end{center}

\end{minipage}
\hfill
\begin{minipage}{0.44\textwidth}
\begin{center}
\includegraphics[width=\textwidth]{\pathToPlots/JTBDT_Spring17v10/wo_NP/PseudoData/r_ttbb_r_ttcc/63445464_ttHbb_N1000_r_ttbb_r_ttcc_ttbarPlusBBbar_1p2_pullplot_MDFnominal_S_0p0.pdf}\\
%\vskip -0.2cm
Pull Plot of MDF for $\mu = 0$
\end{center}
\end{minipage}
\begin{itemize}
\item visible pulls for parameters other than \texttt{bgnorm\_ttbarPlusBBbar} (r-Fit) and r\_ttbb (MDF)\\
\rar expectation: normalizations of postfit process templates is different than expected values
\end{itemize}
\end{frame}

\begin{frame}{Fit to PseudoData - Pull Plots for $\mu = 1$}

\begin{minipage}{0.44\textwidth}
\centering
\includegraphics[width=\textwidth]{\pathToPlots/JTBDT_Spring17v10/wo_NP/PseudoData/r_ttbb_r_ttcc/63445464_ttHbb_N1000_r_ttbb_r_ttcc_ttbarPlusBBbar_1p2_pullplot_nominal_S_1p0.pdf}\\
%\vskip -0.2cm
Pull Plot of r-Fit for $\mu = 1$

\end{minipage}
\hfill
\begin{minipage}{0.44\textwidth}
\centering
\includegraphics[width=\textwidth]{\pathToPlots/JTBDT_Spring17v10/wo_NP/PseudoData/r_ttbb_r_ttcc/63445464_ttHbb_N1000_r_ttbb_r_ttcc_ttbarPlusBBbar_1p2_pullplot_MDFnominal_S_1p0.pdf}\\
%\vskip -0.2cm
Pull Plot of MDF for $\mu = 1$

\end{minipage}
\begin{itemize}
\item pulls in S+B fit less pronounced than in B-Only fit
\item other than that, same behavior as in $\mu = 0$ case
\end{itemize}
\end{frame}

\subsection{Normalizations for j $\geq 6$, t $\geq 4$}

\begin{frame}{Fit to PseudoData - Normalization Pull Plots for j$\geq 6$, t$\geq 4$, $\mu = 0$}

\begin{minipage}{0.44\textwidth}
\centering
\includegraphics[width=\textwidth]{\pathToPlots/JTBDT_Spring17v10/wo_NP/PseudoData/r_ttbb_r_ttcc/63445464_ttHbb_N1000_r_ttbb_r_ttcc_ttbarPlusBBbar_1p2_ljets_jge6_tge4_normalisation_pullplot_nominal_S_0p0.pdf}\\
%\vskip -0.2cm
Pull Plot of r-Fit $\mu = 0$

\end{minipage}
\hfill
\begin{minipage}{0.44\textwidth}
\centering
\includegraphics[width=\textwidth]{\pathToPlots/JTBDT_Spring17v10/wo_NP/PseudoData/r_ttbb_r_ttcc/63445464_ttHbb_N1000_r_ttbb_r_ttcc_ttbarPlusBBbar_1p2_ljets_jge6_tge4_normalisation_pullplot_MDFnominal_S_0p0.pdf}\\
%\vskip -0.2cm
Pull Plot of MDF for $\mu = 0$

\end{minipage}
\begin{itemize}
\item normalizations of signal processes fits expectation
\item background processes are overestimated
\end{itemize}
\end{frame}

\begin{frame}{Fit to PseudoData - Normalization Pull Plots for $\mu = 1$}

\begin{minipage}{0.44\textwidth}
\centering
\includegraphics[width=\textwidth]{\pathToPlots/JTBDT_Spring17v10/wo_NP/PseudoData/r_ttbb_r_ttcc/63445464_ttHbb_N1000_r_ttbb_r_ttcc_ttbarPlusBBbar_1p2_ljets_jge6_tge4_normalisation_pullplot_nominal_S_1p0.pdf}\\
%\vskip -0.2cm
Pull Plot of r-Fit for $\mu = 1$

\end{minipage}
\hfill
\begin{minipage}{0.44\textwidth}
\centering
\includegraphics[width=\textwidth]{\pathToPlots/JTBDT_Spring17v10/wo_NP/PseudoData/r_ttbb_r_ttcc/63445464_ttHbb_N1000_r_ttbb_r_ttcc_ttbarPlusBBbar_1p2_ljets_jge6_tge4_normalisation_pullplot_MDFnominal_S_1p0.pdf}\\
%\vskip -0.2cm
Pull Plot of MDF for $\mu = 1$

\end{minipage}
\begin{itemize}
\item same behavior as in $\mu = 0$ case
\end{itemize}
\end{frame}
